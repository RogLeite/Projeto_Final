% -*- coding: iso-8859-1; -*-

\newcommand{\degree}{\ensuremath{^\circ}}

\newcommand{\myworries}[1]{\textcolor{red}{#1}}
%\renewcommand\myworries[1]{}

\newcommand{\tam}{Transcrição Automática de Música}
\newcommand{\TAM}{TAM}

\newcommand{\hmm}{Hidden Markov Model }
\newcommand{\HMM}{HMM}

\newcommand{\mir}{Music Information Retreival}
\newcommand{\MIR}{MIR}

\newcommand{\ca}{\emph{clave de articulação} }
\newcommand{\CA}{\emph{designação nominativa} }


\newcommand{\cronogramaoriginal}{
% Please add the following required packages to your document preamble:
% \usepackage[table,xcdraw]{xcolor}
% If you use beamer only pass "xcolor=table" option, i.e. \documentclass[xcolor=table]{beamer}
\begin{table}[]
\begin{tabular}{|p{5cm}|l|l|l|l|l|l|l|l|l|}
\hline
Atividade & Out & Nov & Dez & Jan & Fev & Mar & Abr & Mai & Jun \\ \hline
Estudo de tecnologias para interpretação de sinais. & \cellcolor[HTML]{C0C0C0}P &  &  &  &  &  &  &  &  \\
 & \cellcolor[HTML]{34FF34}R & \cellcolor[HTML]{34FF34}R &  &  &  &  &  &  &  \\ \hline
Criar dados de teste: notas individuais &  & \cellcolor[HTML]{C0C0C0}P & \cellcolor[HTML]{C0C0C0}P & \cellcolor[HTML]{C0C0C0}P &  &  &  &  &  \\
 &  & \cellcolor[HTML]{FE0000}N & \cellcolor[HTML]{FE0000}N & \cellcolor[HTML]{C0C0C0}P &  &  &  &  &  \\ \hline
Criar dados de teste: Conjuntos de notas formando um ritmo &  &  &  & \cellcolor[HTML]{C0C0C0}P & \cellcolor[HTML]{C0C0C0}P & \cellcolor[HTML]{C0C0C0}P &  &  &  \\
 &  &  &  & \cellcolor[HTML]{C0C0C0}P & \cellcolor[HTML]{C0C0C0}P & \cellcolor[HTML]{C0C0C0}P &  &  &  \\ \hline
Estudo de tecnologias para interpretação rítmica &  & \cellcolor[HTML]{C0C0C0}P &  &  &  &  &  &  &  \\
 &  & \cellcolor[HTML]{34FF34}R & \cellcolor[HTML]{34FF34}R &  &  &  &  &  &  \\ \hline
Estudo de representações de partituras. &  &  & \cellcolor[HTML]{C0C0C0}P &  &  &  &  &  &  \\
 &  & \cellcolor[HTML]{34FF34}R & \cellcolor[HTML]{34FF34}R &  &  &  &  &  &  \\ \hline
Implementação da interpretação de sinais &  &  & \cellcolor[HTML]{C0C0C0}P & \cellcolor[HTML]{C0C0C0}P &  &  &  &  &  \\
 &  &  & \cellcolor[HTML]{FE0000}N & \cellcolor[HTML]{C0C0C0}P &  &  &  &  &  \\ \hline
Testes da interpretação de sinais &  &  &  & \cellcolor[HTML]{C0C0C0}P &  &  &  &  &  \\
 &  &  &  & \cellcolor[HTML]{C0C0C0}P &  &  &  &  &  \\ \hline
Implementação da interpretação rítmica &  &  &  &  & \cellcolor[HTML]{C0C0C0}P & \cellcolor[HTML]{C0C0C0}P &  &  &  \\
 &  &  &  &  & \cellcolor[HTML]{C0C0C0}P & \cellcolor[HTML]{C0C0C0}P &  &  &  \\ \hline
Testes da interpretação rítmica &  &  &  &  &  & \cellcolor[HTML]{C0C0C0}P &  &  &  \\
 &  &  &  &  &  & \cellcolor[HTML]{C0C0C0}P &  &  &  \\ \hline
Implementação da escrita de partitura. &  &  &  &  &  &  & \cellcolor[HTML]{C0C0C0}P & \cellcolor[HTML]{C0C0C0}P &  \\
 &  &  &  &  &  &  & \cellcolor[HTML]{C0C0C0}P & \cellcolor[HTML]{C0C0C0}P &  \\ \hline
Testes da escrita de partitura &  &  &  &  &  &  &  & \cellcolor[HTML]{C0C0C0}P &  \\
 &  &  &  &  &  &  &  & \cellcolor[HTML]{C0C0C0}P &  \\ \hline
Testes da integração do sistema. &  &  &  &  &  &  &  & \cellcolor[HTML]{C0C0C0}P & \cellcolor[HTML]{C0C0C0}P \\
 &  &  &  &  &  &  &  & \cellcolor[HTML]{C0C0C0}P & \cellcolor[HTML]{C0C0C0}P \\ \hline
\end{tabular}
\caption{Cronograma Planejado/Realizado}
\end{table}
}

\newcommand{\cronogramanovo}
{
% Please add the following required packages to your document preamble:
% \usepackage[table,xcdraw]{xcolor}
% If you use beamer only pass "xcolor=table" option, i.e. \documentclass[xcolor=table]{beamer}
\begin{table}[width=\textwidth]
\begin{tabular}{|l|lllllll}
\hline
Atividade & \multicolumn{1}{l|}{Dez} & \multicolumn{1}{l|}{Jan} & \multicolumn{1}{l|}{Fev} & \multicolumn{1}{l|}{Mar} & \multicolumn{1}{l|}{Abr} & \multicolumn{1}{l|}{Mai} & \multicolumn{1}{l|}{Jun} \\ \hline
Criar dados de teste: notas individuais & \multicolumn{1}{l|}{\cellcolor[HTML]{C0C0C0}P} & \multicolumn{1}{l|}{\cellcolor[HTML]{C0C0C0}P} & \multicolumn{1}{l|}{} & \multicolumn{1}{l|}{} & \multicolumn{1}{l|}{} & \multicolumn{1}{l|}{} & \multicolumn{1}{l|}{} \\ \hline
Implementação da interpretação de sinais & \multicolumn{1}{l|}{} & \multicolumn{1}{l|}{\cellcolor[HTML]{C0C0C0}P} & \multicolumn{1}{l|}{\cellcolor[HTML]{C0C0C0}P} & \multicolumn{1}{l|}{} & \multicolumn{1}{l|}{} & \multicolumn{1}{l|}{} & \multicolumn{1}{l|}{} \\ \hline
Testes da interpretação de sinais & \multicolumn{1}{l|}{} & \multicolumn{1}{l|}{} & \multicolumn{1}{l|}{\cellcolor[HTML]{C0C0C0}P} & \multicolumn{1}{l|}{} & \multicolumn{1}{l|}{} & \multicolumn{1}{l|}{} & \multicolumn{1}{l|}{} \\ \hline
Implementação da interpretação rítmica & \multicolumn{1}{l|}{} & \multicolumn{1}{l|}{} & \multicolumn{1}{l|}{\cellcolor[HTML]{C0C0C0}P} & \multicolumn{1}{l|}{\cellcolor[HTML]{C0C0C0}P} & \multicolumn{1}{l|}{\cellcolor[HTML]{C0C0C0}P} & \multicolumn{1}{l|}{} & \multicolumn{1}{l|}{} \\ \hline
Testes da interpretação rítmica & \multicolumn{1}{l|}{} & \multicolumn{1}{l|}{} & \multicolumn{1}{l|}{} & \multicolumn{1}{l|}{\cellcolor[HTML]{C0C0C0}P} & \multicolumn{1}{l|}{\cellcolor[HTML]{C0C0C0}P} & \multicolumn{1}{l|}{} & \multicolumn{1}{l|}{} \\ \hline
Implementação da escrita de partitura. & \multicolumn{1}{l|}{} & \multicolumn{1}{l|}{} & \multicolumn{1}{l|}{} & \multicolumn{1}{l|}{} & \multicolumn{1}{l|}{\cellcolor[HTML]{C0C0C0}P} & \multicolumn{1}{l|}{\cellcolor[HTML]{C0C0C0}P} & \multicolumn{1}{l|}{} \\ \hline
Testes da escrita de partitura & \multicolumn{1}{l|}{} & \multicolumn{1}{l|}{} & \multicolumn{1}{l|}{} & \multicolumn{1}{l|}{} & \multicolumn{1}{l|}{} & \multicolumn{1}{l|}{\cellcolor[HTML]{C0C0C0}P} & \multicolumn{1}{l|}{} \\ \hline
Testes da integração do sistema. & \multicolumn{1}{l|}{} & \multicolumn{1}{l|}{} & \multicolumn{1}{l|}{} & \multicolumn{1}{l|}{} & \multicolumn{1}{l|}{} & \multicolumn{1}{l|}{\cellcolor[HTML]{C0C0C0}P} & \multicolumn{1}{l|}{\cellcolor[HTML]{C0C0C0}P} \\ \hline
\cellcolor[HTML]{C0C0C0}P - Planejado &  &  &  &  &  &  &  \\ \cline{1-1}
\end{tabular}
\caption{Cronograma para desenvolvimento no Projeto Final II}
\end{table}
}


%%%
%%%
%%%
\newcommand{\mybulletOB}{%
  % \textbullet
  % \checkmark
  $\triangleright$
  %\textopenbullet
}

\newcolumntype{L}{>{\raggedright \arraybackslash}X}
\newcolumntype{R}{>{\raggedleft \arraybackslash}X}
\newcolumntype{C}{>{\centering \arraybackslash}X}
\newcolumntype{M}[1]{>{\centering\hspace{0pt}}m{#1}}

\newcommand{\mrcel}[2]{%
\begin{tabular}[c]{@{}c@{}}#1\\#2\end{tabular}}

\newcommand{\mrcell}[2]{%
\begin{tabular}[l]{@{}l@{}}#1\\#2\end{tabular}}

\newcommand{\mrcelthree}[3]{%
\begin{tabular}[c]{@{}c@{}c@{}}#1\\#2\\#3\end{tabular}}

\newcommand{\mrcelcolorg}[2]{%
\begin{tabular}{l}\rowcolor{Gainsboro}#1\\#2\end{tabular}}

\newcommand{\tbthreeblabla}[3]{%
\begin{tabular}{ l @{\extracolsep{2mm}}X }
  \mybulletOB #1
  \mybulletOB #2
  \mybulletOB #3
\end{tabular}}

\newcommand{\mytbcimg}[3]{%
  \multicolumn{1}{C}{\parbox[c]{#1}{\includegraphics[width=#2]{#3}}}}

%%%
%%% In table-2.1
%%%

\newcommand{\colmund}[1]{\npmakebox[abcdef][c]{#1\degree}}